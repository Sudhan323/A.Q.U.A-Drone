\chapter{APPLICATION OF UNDERWATER DRONE}\thispagestyle{EmptyHeader}
\label{chp:1}

The A.Q.U.A Drone can be used for many applications such as Fish Detection, Underwater pipe line detection, Ship Hull inspection, Dam inspection and so on. Two of these application have been implemented in this project.

\section{Fish Detection}
Underwater drones are used for many purposes. They have wide range of applications in them. One of them is Fish Detection in underwater. It is difficult for a person to find where there will be high population of fishes. For that reason the fish detection application can be used.

The drone captures the video from the sea surface which will consist of many fishes or other marine life forms. They can be easily identified or detected using the fish detection method. This program detects a fish in an video and draws a rectangle box around it. This is done so that it makes easy for the user to identify the fish present in the video.

The further applications of fish detection is identification of fishes present in them. That is the program can be further developed into a deep learning model to categorise the fishes that is been detected in the video to different species or different varieties.

For this python and opencv software is used. A cascade filter for the fish detection is been created using the previously available images and a cascade filter is been created using the database. After creation of the cascade filter the python program is developed to identify the fishes available in an video and draw a rectangular box around them. This makes the user to easily locate the fishes.

\section{Crack Detection in Underwater Pipeline}
There are many international underwater pipeline which carries highly valuable things in them. These pipes are subjected to many possible ways of damage. They are highly vulnerable to damage. The wild life forms under the water can cause severe damage to them. These are needed to be monitored regularly. Therefore we have developed an program to find the crack present in the underwater pipeline and calculate the area of the crack and length of the crack.

\begin{figure}[ht]
	\centering
	\includegraphics[width=\linewidth]{images/CrackDetection.jpg}
	\caption{Crack Detection in pipeline}
	\label{diag:sample}
\end{figure}

The way of approach behind the idea is too simple. The idea is to find different contours of the defected pipeline and draw them. The cracks present in the pipe will always be in black colour which can be identified in the python program and area as well as the length of the contour are found and converted into milli-meter square. The initial requirement of the program is the diameter of the pipe. The diameter of the pipe has to be mentioned to find the area of the crack.

\clearpage
\newpage