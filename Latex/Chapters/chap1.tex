%% This is an example first chapter.  You should put chapter/appendix that you
%% write into a separate file, and add a line \include{yourfilename} to
%% main.tex, where `yourfilename.tex' is the name of the chapter/appendix file.
%% You can process specific files by typing their names in at the
%% \files=
%% prompt when you run the file main.tex through LaTeX.
\chapter{INTRODUCTION}\thispagestyle{EmptyHeader}
\label{chp:1}

The project Aquatic drone is based upon the domain "life under water". The project is to monitor the underwater surface through the camera installed in the drone. The drone is controlled by the user from the surface. Live feed of the camera is given to the used through cable connected from the drone.

\section{Background}
70 percent of the earths surface is covered by water. But only 30 percent of these surface is explored. The remaining is till to be a myth for humans. To study about various surface of the earth's crust it is very important to study about the sea surfaces. Humans can reach only limited dept of the sea using the scuba diving equipment. But to reach higher dept there is need for a drone. The drone is used to monitor the underwater surface using a high performance camera. This camera takes pictures or gives a live video feed to the user in the surface. The underwater drones can be used to monitor various things such as pipelines, ships, ships wreaks, marine life and so on. The development of these drones in India is still under development.

\section{Motivation}
Underwater exploration is still a major achievement for us to understand the deep sea floors of our planet. Many researches are in still progress to explore the sea bed of our oceans and seas. Underwater exploration is very important for us because in recent years many waste has been dumped in the oceans and these waste have affected many living organisms. The survival of every living being is important for the balance of eco-system. Underwater inspection will also give the detail of population of fishes and other sea food, which makes easier for the fisher-men to know the exact location of the fish. Still in many places underwater exploration is done manually using scuba diving. But there are some limitations for that such as high pressure and lack of oxygen. To overcome this problem autonomous drone or remotely controlled drone can be used. These drones are much faster and more efficient then the scuba divers.

\section{Problem Statement}
Need of drones for underwater investigation such as fish monitoring, Ship hull monitoring and Pipeline Monitoring.

\section{Market Research}
Underwater drones can be broadly classified into automatic drones and remotely controlled drones (ROV). Under ROV there are many classes such as Work class, Light Work class, Observation class and Micro class ROVs. The different classes have different purposes such as monitoring, make repairs of ships and water testing. The usage of ROVs in India is still under implementation. The ROV manufacturing in India is very less as compared to other foreign nations. Still many works such as underwater pipeline inspection, underwater hull inspection and marine life monitoring is done manually. Many new startups have developed these kinds of underwater drones and are made for sale. Still there is a lack of small scale ROVs for the usage of aquaculture. A large ROV can be used for monitoring of fishes underwater but it may affect their living nature. To avoid this micro ROVs are used for aquaculture.

\section{Feasibility}
The proposed idea is very useful for the small scale users of ROVs such as fishermen. Since the size of the ROV is small when compared to the ROVs that is available in the market this can be easily controlled and operated for simple operations. Task such as monitoring of population of fishes in a certain area can be done very easily. The application of this ROV can be extended in many ways such as inspection of underwater pipeline, hull inspection of a ship and dam inspections. The number of propellers used is only four which saves most of the energy and the working time under water is estimated to be 3 hours. This ROV comes under micro class ROV and are very easy to operate. The live video feed is the most important part of the ROV. Due to the cheaper cost, easy installation, maintenance and environmental friendly the project is hundred percent feasible in the real time scenario.

\clearpage
\newpage